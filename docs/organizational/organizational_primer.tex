\documentclass{article}

\usepackage{hyperref}
\title{Primer on the organizer role in the COMP354 PK-B team}
\date{2020-01-18}
\author{Pavel Balan}
\begin{document}
\maketitle
\pagenumbering{gobble}
\newpage

\section{I'm an organizer. What do I do?}


First and foremost, welcome. This position (as I am finding) is an interesting one. Don't expect to write much code at all, unless coding team needs extra hands. Odd for a CompSci class, eh?


To make sure you know what to do (to make sure -I- know what to do, even), I have tried compiling a list of things I believe the team would find handy.


\subsection{Manage meetings}


Meetings happen. You go to the lab, you talk about some things. Important things! Then you forget those things. It is useful to keep a written record of what exactly was talked about, so that the team does not end up going in circles too much.


There is a google doc we keep on hand for managing minutes (minutae?) of past meetings - please write what was agreed upon every time the team comes together.


Remind people of it a couple days before the lab, also. Ask them to fill the agenda, fill the agenda yourself too - the team should know what to talk about during a meeting. (If you have no goal, how productive can you really be?)


The biggest plus of meeting notes is that you can put stuff from there into your diaries easy-peasy. Gives your team (and you personally!) a better grade.


\subsection{Enforce policy}


We decide on a lot of things. That's how big software works - nine people are making the same game, and the hardest part is to swim in the same direction as everyone else.


There's no magic bolt of lightning that will smite a person every time they don't do as was agreed upon. Left unchecked, swimming in the same direction will become harder and harder.


If people agreed on something during a meeting, it's not the coder's or the documenter's duty to nag others to do as the agreement said. That means it's your job.


Look at previous meeting notes, make sure whatever is written in them was not written to be tossed aside and quietly discarded. Nag and pester are your sword and shield here.


\subsection{Keep the shared spaces clean}


We have a Github repository. What happens in the code folder is the code team's responsibility, what happens in the docs folder is the docs team's responsibility, what happens in the root folder is your responsibility. There is also a Trello board, but I have not adopted the technology enough to know my way around it. I'll update it once we begin using it. If you're the organizer and it's not updated, go kick me (Pavel) until I do.

\subsection{Communicate with the professor}


I trust our coders to make the best code we can make. I trust our documenters to create the best documentation we can make. We also have a grading scheme. Going over the entire repository without thinking "good" or "bad", and just looking at the grading rubrics is probably what the professor is going to do. We should also do that, to avoid nasty surprises. Please take this responsibility as the organizer.


Also, organize the submission effort.


\subsection{Snacks}


(Optional, but heavily encouraged)


\subsection{Most importantly, trust your team}


We all want a good grade. We'll do what we can to make sure it's not bad. Make sure we are not stuck, make sure we are not totally confused. If everything is fine, we'll write good code and do good docs.


\newpage

\section{Links and resources for the practical organizer}


\href{https://docs.google.com/document/d/1XMhbWeMdVgvgZoUz4VCezFH5bnMrgQWkQ1Zt4rR_wwE/edit}{Our meeting agenda/notes doc}

\href{https://comp354pk-b.slack.com/}{Our Slack channel}

\href{https://tex.stackexchange.com/questions/183358/how-to-convert-a-google-docs-document-to-latex}{How to convert GDocs to LaTeX (FYI)}


\end{document}
