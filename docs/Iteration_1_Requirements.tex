\documentclass[12pt]{article}

\pagestyle{empty}
\setcounter{secnumdepth}{2}

\topmargin=0cm
\oddsidemargin=0cm
\textheight=22.0cm
\textwidth=16cm
\parindent=0cm
\parskip=0.15cm
\topskip=0truecm
\raggedbottom
\abovedisplayskip=3mm
\belowdisplayskip=3mm
\abovedisplayshortskip=0mm
\belowdisplayshortskip=2mm
\normalbaselineskip=12pt
\normalbaselines

\begin{document}

\vspace*{0.5in}
\centerline{\bf\Large Requirements Document}

\vspace*{0.5in}
\centerline{\bf\Large Team PK-B}

\vspace*{0.5in}
\centerline{\bf\Large 17 January 2020}

\vspace*{1.5in}
\begin{table}[htbp]
\caption{Team Members}
\begin{center}
\begin{tabular}{|l | c|}
\hline
Name & Student ID \\
\hline\hline
Pavel Balan & ID Number \\ \hline
Ivan Garzon & 27006284 \\ \hline
Ki Ho Lee & ID Number \\ \hline
Nian Lui & ID Number \\ \hline
Anthony Menard-Gill & 40050437 \\ \hline
Micheal Naccache & ID Number \\ \hline
Ashish Saha & ID Number \\ \hline
Nyriam Tayah & ID Number \\ \hline
Mona Shayvard & ID Number \\ \hline
\end{tabular}
\end{center}
\end{table}

\clearpage

\section{System}

\subsection{Purpose}

\subsection{Context}

\subsection{Business Goals}

\section{Domain Concepts}

\section{Actors}

\section{Use Cases}

\subsection{Overview}
\pagebreak
\begin{figure}[htbp]
%insert diagram here
\caption{Use Case Diagram}
\label{fig:use-case-diagram}
\end{figure}

\subsubsection{Use Case 1} \label{uc:1}
\begin{tabular}{|p{0.25\textwidth}|p{0.75\textwidth}|}
\hline
\bf Name & Start Game \\ \hline
\bf Summary & The user press the button Start Game. A 10x10 grid is displayed  on the left side with the leftmost column with black squares and the topmost row with black squares. On the right side, there will be a list of rules describing the game. The clues will be displayed in the black cells. \\ \hline
\bf Actors & Player \\ \hline
\bf Preconditions & 
\begin{itemize}
    \item Choosing the level of difficulty. If the user does not choose a level, the default 'easy' level will be applied.
    \item The user can enter a username before starting the game, if it is not chosen the default username is: user1. 
\end{itemize} \\ \hline
\bf Main Scenario & 
\begin{enumerate}
  \item The player clicks on the Start Game button.
  \item The grid is initialized on the left side with clues and empty cells.
  \item The numeric keypad with numbers 1 to 9 is displayed on the right side.
  \item The game menu buttons are displayed on the right side.
  \item The rules of the game are displayed on the right side.
\end{enumerate} \\ \hline
\bf Exceptions & 
\begin{itemize}
    \item The player clicks on exit to end the game.
    \item The player chooses a different level and the grid is re-drawn.
    \item The player decides to enter his username and the interface takes him to the appropriate window.
\end{itemize} \\ \hline
\bf Postcondition & None \\ \hline
\bf Priority & The game is in play mode and the options displayed belong only to the current game. \\ \hline
\bf Traces to Test Case & The start game case is done after clicking on the button. \\ \hline
\end{tabular}

\subsubsection{Use Case 2} \label{uc:2}
\begin{tabular}{|p{0.25\textwidth}|p{0.75\textwidth}|}
\hline
\bf Name & Give a name \\ \hline
\bf Summary & Short summary/description/story \\ \hline
\bf Actors & List of actors \\ \hline
\bf Precondition & List of preconditions \\ \hline
\bf Main Scenario & Describe step 1, 2, 3, etc. \\ \hline
\bf Exceptions & List exceptions \\ \hline
\bf Postcondition & List postconditions \\ \hline
\bf Priority & Note priority \\ \hline
\bf Traces to Test Case & Add when test cases done \\ \hline
\end{tabular}

\section{Non-Functional Constraints}

\section{Data Dictionary}

\section{References}

\appendix

\section{Description of File Format: Tasks}

Describe input file format.

\section{Description of File Format: Persons}

Describe output file format.

\end{document}
