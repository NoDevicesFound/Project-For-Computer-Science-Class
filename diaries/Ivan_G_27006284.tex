\documentclass[12pt]{article}

\pagestyle{empty}
\setcounter{secnumdepth}{0}

\topmargin=0cm
\oddsidemargin=0cm
\textheight=22.0cm
\textwidth=16cm
\parindent=0cm
\parskip=0.15cm
\topskip=0truecm
\raggedbottom
\abovedisplayskip=3mm
\belowdisplayskip=3mm
\abovedisplayshortskip=0mm
\belowdisplayshortskip=2mm
\normalbaselineskip=12pt
\normalbaselines

\begin{document}

\vspace*{0.2in}
\centerline{\bf\Large Diary}

\vspace*{0.2in}
\centerline{\bf\Large Name: Ivan Garzon   Student ID: 27006284}

\vspace*{0.2in}
\centerline{\bf\Large Team PK-B}

\vspace*{0.2in}
\centerline{\bf\Large January-April, 2019}

\section{Iteration 1}

{\bf Date:} January 10, 2020\\
{\bf Start Time:} 18:15\\
{\bf End Time:} 19:00 \\
{\bf Who:} Ivan Garzon \\
{\bf Where:} Library \\
{\bf Activities:} Downloaded project pdf and read project description and expected features. Read team organization, roles and project stages. Analyzed project plan schedule, iterations and deliverables.\\
{\bf Outcomes:} Learned about the project, and the goals expected at every stage. Analyzed carefully what software needs to be used to develop the source code and documentation. Learned about each role and how it will be designated through the project duration. Went over each iteration gathering important information to keep in mind for the upcoming tasks.\\\\

{\bf Date:} January 14, 2020\\
{\bf Start Time:} 16:30\\
{\bf End Time:} 17:30\\
{\bf Who:} Ivan Garzon\\
{\bf Where:} Library\\
{\bf Activities:} Read the information about Kakuro game in Wikipedia and played 1 game from the website given in the project description. Wrote all the rules about the game in a notepad. Viewed several Kakuro GUI already existing on the web to get some ideas and inspiration.\\
{\bf Outcomes:} Learned the basic rules of the game. Took note of additional options to have in our version of the game. Visualize designed and analyzed the game implementation from development perspective.\\\\

{\bf Date:} January 15, 2020\\
{\bf Start Time:} 19:15\\
{\bf End Time:} 20:45\\
{\bf Who:} All the team members\\
{\bf Where:} H:831\\
{\bf Activities:} The team assisted to the first lab of this semester. We had the chance to have a small team meeting and discuss topics such as: semester schedule, team members who work and their time constraints, roles, tasks, git.\\
{\bf Outcomes:} We chose the roles of each team member resulting into 5 documenters, 3 coders, 1 organizer. Since the first iteration is heavily concentrated we decided to have more team members as documenters than for the other roles. There is flexibility of adding one member to the documenters, depending on time constraints. Pavel created minutes doc and is assigned the task to create a github repository. As a team orginazer he will coordinate all team activities. Ashish was selected as the person who can transfer LaTex knowledge acquired previously. Myriam will give a JUnit tutorial to the developers. Ivan will create a Trello board as a project planning platform. We are all given the task to self learn LaTex and start to write our first diary entries.\\\\

{\bf Date:} January 16, 2020\\
{\bf Start Time:} 20:30\\
{\bf End Time:} 21:20\\
{\bf Who:} All the team members\\
{\bf Where:} H:523\\
{\bf Activities:} The team assisted to the first tutorial of this semester. We had the chance to discuss the use cases of the game, and how to implement and coordinate with the development team \\
{\bf Outcomes:} We concluded that since the development team need some time to learn the technologies and code practices, the documenters would begin creating the use cases independently of the developers. And as the documenters start to create their uses cases, they would consult with the developers continuously and adjust the use cases based on design and/or development constraints.\\\\

{\bf Date:} January 18, 2020\\
{\bf Start Time:} 11:00\\
{\bf End Time:} 12:30\\
{\bf Who:} Ivan Garzon\\
{\bf Where:} Concordia Library\\
{\bf Activities:} Watch youtube tutorials on LatEx and practice. Create Trello board for the team, create new cards and add description to each task. Added team members to the board. \\
{\bf Outcomes:} Learned the basic functionality of LatEx. Was able to create a Trello board, adding lists of activity cards and creating project tasks for each team.\\\\

%\section{Iteration 2}

%\section{Iteration 3}

\end{document}
