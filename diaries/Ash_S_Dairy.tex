\documentclass[12pt]{article}

\pagestyle{empty}
\setcounter{secnumdepth}{0}

\topmargin=0cm
\oddsidemargin=0cm
\textheight=22.0cm
\textwidth=16cm
\parindent=0cm
\parskip=0.15cm
\topskip=0truecm
\raggedbottom
\abovedisplayskip=3mm
\belowdisplayskip=3mm
\abovedisplayshortskip=0mm
\belowdisplayshortskip=2mm
\normalbaselineskip=12pt
\normalbaselines

\begin{document}

\vspace*{0.2in}
\centerline{\bf\Large Diary}

\vspace*{0.2in}
\centerline{\bf\Large Name: Ashish Saha.  Student ID: 27549814.}

\vspace*{0.2in}
\centerline{\bf\Large Team PK-B}

\vspace*{0.2in}
\centerline{\bf\Large 08 January 2019}

\section{Iteration 1}

{\bf Date:} 08 January 2020. \\
{\bf Start Time:} 4:00PM. \\
{\bf End Time:} 5:00PM. \\
{\bf Duration:} 60 minutes. \\
{\bf Who:} Ashish Saha. \\
{\bf Where:} Concordia Webster Library. \\
{\bf Activities:} Went through the professor's class website where I familiarized with the project description, saw an initial view of the team, and set up my Eclipse development environment alongside with JavaFX and Java's SceneBuilder.  \\
{\bf Outcomes:} Eclipse development environment was successfully set up, and a JavaFX test project was able to be compiled.
\vspace*{0.2in} \\
{\bf Date:} 09 January 2020. \\
{\bf Start Time:} 8:15PM. \\
{\bf End Time:} 8:30PM. \\
{\bf Duration:} 15 minutes. \\
{\bf Who:} Pavel Balan, Ivan Garzon, Ki Ho Lee, Nian Liu, Anthony Menard-Gill, \\
\hspace{10mm} Micheal Naccache, Ashish Saha, Myriam Tayah, Shuo Zhang. \\
{\bf Where:} Corridor outside room H-920 \\
{\bf Activities:} We first acquainted with all present team members. Then, we decided on which platform to use between messenger, discord or slack. Furthermore, we gathered initial information on who was familiar with LaTeX. \\
{\bf Outcomes:} Initial team members were successfully acquainted, a Slack room and a GitHub repository was created, and invitations were sent out by Pavel Balan.
\pagebreak \\
{\bf Date:} 15 January 2020. \\
{\bf Start Time:} 7:15PM. \\
{\bf End Time:} 8:45PM. \\
{\bf Duration:} 90 minutes. \\
{\bf Who:} Pavel Balan, Ivan Garzon, Ki Ho Lee, Nian Liu, Anthony Menard-Gill, \\
\hspace{10mm} Micheal Naccache, Ashish Saha, Myriam Tayah, Mona Shayvard. \\
{\bf Where:} Room H-831. \\
{\bf Activities:} We first acquainted with our newest team member Mona Shayvard. Then, we decided on role assignments for the first iteration. After that, we discussed assigning a use case per team member, of which the three most important will be chosen for the first iteration. I was asked to guide the documenters through using LaTeX, Myriam Tayah proposed hosting a JUnit tutorial for the coders to be done, and Ivan Garzon coined the setup of a Trello board in order to keep track of team progress. We also brainstormed ideas on what could be a first draft on use cases for the Kakuro game.\\
{\bf Outcomes:} Mona was added to the Slack room; Pavel Balan was assigned role of Organizer; 
Ivan Garzon, Anthony Menard-Gill, Micheal Naccache, myself and Mona Shayvard were assigned the role of Documenters; Ki Ho Lee, Nian Liu and Myriam Tayah were assigned roles of Coders; a JUnit tutorial session was set up amongst the Coder team; and an initial list of use cases were created.
\vspace*{0.2in} \\
{\bf Date:} 16 January 2020. \\
{\bf Start Time:} 8:30PM. \\
{\bf End Time:} 9:20PM. \\
{\bf Duration:} 50 minutes. \\
{\bf Who:} Pavel Balan, Ivan Garzon, Ki Ho Lee, Nian Liu, Anthony Menard-Gill, \\
\hspace{10mm} Micheal Naccache, Ashish Saha, Myriam Tayah, Mona Shayvard. \\
{\bf Where:} Room H-523. \\
{\bf Activities:} We learned about use cases diagrams and descriptions, along with domain models from the Teaching Assistant (TA). The TA then set up an exercise for teams to come up with sample use cases for the Kakuro game which allowed us to expand on our initial draft list. \\
{\bf Outcomes:} Knowledge was gained on coming up with relevant use cases, and domain models; and our use case list was successfully expanded upon.

%\section{Iteration 2}

%\section{Iteration 3}

\end{document}
