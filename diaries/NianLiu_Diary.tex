\documentclass[12pt]{article}

\pagestyle{empty}
\setcounter{secnumdepth}{0}

\topmargin=0cm
\oddsidemargin=0cm
\textheight=22.0cm
\textwidth=16cm
\parindent=0cm
\parskip=0.15cm
\topskip=0truecm
\raggedbottom
\abovedisplayskip=3mm
\belowdisplayskip=3mm
\abovedisplayshortskip=0mm
\belowdisplayshortskip=2mm
\normalbaselineskip=12pt
\normalbaselines

\begin{document}

\vspace*{0.2in}
\centerline{\bf\Large Diary}

\vspace*{0.2in}
\centerline{\bf\Large Name: Nian Liu   Student ID: 40044346}

\vspace*{0.2in}
\centerline{\bf\Large Team PK-B}

\vspace*{0.2in}
\centerline{\bf\Large 10 January 2019}

\section{Iteration 1}

{\bf Date:} January 9th 2020\\
{\bf Start Time:} 8:00 pm\\
{\bf End Time:} 8:15 pm \\
{\bf Who:} Pavel Balan, Yushun Cheng, Ivan Garzon, Ki Ho Lee, Nian Liu, Anthony Menard-Gill, Michael Naccache, Ashish Saha, Myriam Tayah, Shuo Zhang\\
{\bf Where:} Outside of H-920 \\
{\bf Activities:} The class was end earlier by professor Butler in order to let our team members to have a short meeting and try to know each other. We exchanged our emails and briefly introduced ourselves to each other \\
{\bf Outcomes:} Pavel was going to create a Slack channel for the team after the short meeting. Then we met on the channel to discuss the project further\\

{\bf Date:} January 15th 2020\\
{\bf Start Time:} 7:00 pm\\
{\bf End Time:} 8:55 pm\\
{\bf Who:} Pavel Balan, Ivan Garzon, Ki Ho Lee, Nian Liu, Anthony Menard-Gill, Michael Naccache, Ashish Saha, Myriam Tayah, Mona Shayvard\\
{\bf Where:} H-831 \\
{\bf Activities:}  In the second half of the lab, we had our first formal group meeting hosted by Pavel. Firstly, we meet our new team member Mona. Then, through discussion we did a more detailed division of work according to the project requirements. During the meeting, I sought Mona's advice on how to use LaTex, she gave me a brief introduction of LaTex's application method. I exchanged some ideas aobut the project with everyone in the group. At the end of the meeting, Ki Ho, Myriam and I formed the code group. We set up a small meeting for studying how to use Junit, because Myriam is very experienced in this field, therefore, Ki Ho and I asked Myriam to give us a Junit Tutorial\\
{\bf Outcomes:} After the lab meeting, Ki Ho, Mriam and I became coder group, we set up a small tutorial aobut learning Junit on Thursday afternoon this week\\

{\bf Date:} January 16th 2020\\
{\bf Start Time:} 1:30 pm\\
{\bf End Time:} 3:40 pm \\
{\bf Who:} Ki Ho Lee, Nian Liu, Michael Naccache, Myriam Tayah\\
{\bf Where:} H-831 \\
{\bf Activities:} I introduced the others to a few concepts of unit testing with JUnit. I also explained the difference between unit tests and integration tests. I used a Calculator class I found online and we played around with it. Most importantly, we looked at how to set up unit tests with JUnit in Eclipse and a few industry standards. I did the tutorial twice because Michael joined us a bit after our first try. I asked them a few questions to make sure they were comfortable with everything. We also spent a little bit of time discussing our impressions with the project and got to further know each other at the end. 
At 2:00 pm, Ki Ho and I met Myriam, then she started to teach us how to use Junit to do the testing of code. She had downloaded a Calculator class as an example, to show us how to use Junit step by step. She is very patient and nice. She would show us things we did not understand for several times. Later, Michael joined us. And Myriam did the tutorial again for him, and the same time, Ki Ho and I also review the uses of Junit. After the tutorial, we had a discussion about our working progress of our project. \\
{\bf Outcomes:} I had learned a lot new knowledge about Junit. And I decided to do a little practice when I went home. Ki Ho, Myriam and I also decided to meet up on monday to discuss and learn JavaFX together\\

{\bf Date:} January 18th 2020\\
{\bf Start Time:} 7:00 pm\\
{\bf End Time:} 12:35 pm \\
{\bf Who:} Nian Liu \\
{\bf Where:} At home \\
{\bf Activities:} I watched a 5 hours long tutorial on YouTube about Java Swing. I also wrote a demo game frame by using swing while I was watching the video\\
{\bf Outcomes:} I had learned the basic uses of the GUI javaSwing. However, I found that those are far not enough to finish the project. But the more code I had learned, the more I felt comfortable when I did the Project\\

{\bf Date:} January 19th 2020\\
{\bf Start Time:} 8:00 pm\\
{\bf End Time:} 12:00 pm \\
{\bf Who:} Nian Liu \\
{\bf Where:} At home \\
{\bf Activities:} I started to watch another tutorial video about JavaFX on YouTube, and I tried to rewrite the game frame demo by using JavaFX \\ 
{\bf Outcomes:} I found out that JavaFX and JavaSwing are not very different. Their libraries are very similar. Because I had learn JavaSwing yesterday, I felt I could easier understand the JavaFX. However, my eclipse could not run JavaFX, I spent 2 hours but still could not fix that problem. So I changed to use IntelliJ IDEA to run the JavaFX, but I failed at the end of the day, the running environment for JavaFX was very hard to set up\\

{\bf Date:} January 20th 2020\\
{\bf Start Time:} 2:45 pm\\
{\bf End Time:} 3:55 pm \\
{\bf Who:} Ki Ho Lee, Nian Liu, Ashish Saha, Myriam Tayah, Pavel\\
{\bf Where:} H-841 \\
{\bf Activities:} Ki Ho and I had same class and finish at the same time, so we went to meet Myriam for the meeting to discuss how to use JavaFX to do our Project. When we met Myriam, she just set up a Trello board and invite Ki Ho and I to join to that board. Then we exchanged the information about JavaFX that we had learn these days. And we defined our division of work again. To my surprise, Pavel came to say hi and brought us a lots snacks. A big 'thank you' for our friend, Pavel. What is more, Ashish also came and gave us a tutorial about how to install JavaFX. Through discussion we finally agreed to use JavaFX to do the Project, because other team member are all familiar with using JavaFX. So we think that we should use JavaFX at beginning, then later it would be convenient for other team members when they take over to continue work on our code\\
{\bf Outcomes:} We assigned the tasks for each coder group members on Trello. We took a brief look at the JavaFX installation process. After a few meeting, I found everyone is very friendly and is willing to give help. This inspired me to continue to better accomplish my task\\

%\section{Iteration 2}

%\section{Iteration 3}

\end{document}
